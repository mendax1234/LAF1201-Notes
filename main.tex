%%%%%%%%%%%%%%%%%%%%%%%%%%%%%%%%%%%%%%%%%
% Dictionary
% LaTeX Template
% Version 1.0 (20/12/14)
%
% This template has been downloaded from:
% http://www.LaTeXTemplates.com
%
% Original author:
% Vel (vel@latextemplates.com) inspired by a template by Marc Lavaud
%
% License:
% CC BY-NC-SA 3.0 (http://creativecommons.org/licenses/by-nc-sa/3.0/)
%
%%%%%%%%%%%%%%%%%%%%%%%%%%%%%%%%%%%%%%%%%

%----------------------------------------------------------------------------------------
%	PACKAGES AND OTHER DOCUMENT CONFIGURATIONS
%----------------------------------------------------------------------------------------

\documentclass[10pt,a4paper,twoside]{article} % 10pt font size, A4 paper and two-sided margins

\usepackage[top=3.5cm,bottom=3.5cm,left=3.7cm,right=4.7cm,columnsep=30pt]{geometry} % Document margins and spacings

\usepackage[utf8]{inputenc} % Required for inputting international characters
\usepackage[T1]{fontenc} % Output font encoding for international characters

\usepackage{palatino} % Use the Palatino font

\usepackage{microtype} % Improves spacing

\usepackage{multicol} % Required for splitting text into multiple columns

\usepackage{xcolor} % Required for colors using in delimeter
\usepackage{amsmath}

\usepackage{ tipa } % Required for prononciation

\usepackage[bf,sf,center]{titlesec} % Required for modifying section titles - bold, sans-serif, centered
\usepackage[labelfont=bf, labelsep=none]{caption}

\usepackage{fancyhdr} % Required for modifying headers and footers
\fancyhead[L]{\textsf{\rightmark}} % Top left header
\fancyhead[R]{\textsf{\leftmark}} % Top right header
\renewcommand{\headrulewidth}{1.4pt} % Rule under the header
\fancyfoot[C]{\textbf{\textsf{\thepage}}} % Bottom center footer
\renewcommand{\footrulewidth}{1.4pt} % Rule under the footer
\pagestyle{fancy} % Use the custom headers and footers throughout the document

\newcommand{\entry}[5]{\markboth{#1}{#1}\textbf{#1}\ {(#2)}\ \textit{#3}\ $\bullet$\ \textbf{\textit{#4}} \begin{small}#5\end{small}}  % Defines the command to print each word on the page, \markboth{}{} prints the first word on the page in the top left header and the last word in the top right

%French-specific commands
%--------------------------------------
\usepackage[french]{babel}
\usepackage[autolanguage]{numprint} % for the \nombre command

% Chinese-specific commands
% --------------------------------------
\usepackage{xeCJK} % Required for Chinese characters
\setCJKmainfont{Noto Sans CJK SC} % Use Noto Sans CJK SC instead of SimSun
\usepackage{fontspec} % For handling other fonts like Palatino

%----------------------------------------------------------------------------------------

% Hyper Link Setup
% --------------------------------------
\usepackage{hyperref} % Required for creating hyperlinks
\hypersetup{
    colorlinks=true,
    linkcolor=blue,
    filecolor=magenta,      
    urlcolor=cyan,
    pdftitle={Overleaf Example},
    pdfpagemode=FullScreen,
    }

%----------------------------------------------------------------------------------------

\begin{document}

%----------------------------------------------------------------------------------------
%	Interrogatif
%----------------------------------------------------------------------------------------
\section*{Interrogatif}

\subsection*{Pronom Interrogatif (Unit 1)}
\begin{multicols}{2}
    \entry{où}{}{}{Where}{}

    \entry{quand}{}{}{When}{}

    \entry{comment}{}{}{How}{}

    \entry{quoi}{}{}{What}{}

    \entry{combien}{}{}{How many}{}

    \entry{qui}{}{}{Who}{}
\end{multicols}

\subsection*{Adjective Interrogatif (Unit 1)}
\begin{center}
    Adjective Interrogatif are used at the beginning of a sentence, before the \textbf{noun} they modify
\end{center}
\begin{multicols}{2}
    \entry{quel}{}{}{which, what}{| used when noun \textbf{masculin and singulier} | e.g. Quel est ton nom? Je m'appelle Fernando.}

    \entry{quels}{}{}{which, what}{| used when noun \textbf{masculin and pluriel} | e.g. Tu connais quels pays? Je connais le Canada.}

    \entry{quelle}{}{}{which, what}{| used when noun \textbf{féminin and singulier} | e.g. Quelle est ta nationalité? Je suis chinois.}

    \entry{quelles}{}{}{which, what}{| used when noun \textbf{féminin and pluriel} | e.g. Tu parles quelles langues? Je parle français et espagnol.}
\end{multicols}

%----------------------------------------------------------------------------------------
%	Pronom
%----------------------------------------------------------------------------------------
\section*{Pronom}
\subsection*{Pronom Personnel (Unit 1)}
\begin{multicols}{2}
    \entry{Je}{}{}{I}{}

    \entry{Tu}{}{}{You}{}

    \entry{Il/Elle/On}{}{}{He/She/We}{}

    \entry{nous}{}{}{We}{}

    \entry{vous}{}{}{You}{}

    \entry{ils/elles}{}{}{they}{}

\end{multicols}

\subsection*{Disjunctive Pronoun (Unit 1)}
\begin{center}
    They are used for emphasis or in specific grammatical contexts, such as after prepositions or for contrast.
\end{center}
\begin{multicols}{2}
    \entry{Moi}{}{}{}{Moi, je m'appelle Lisa.}

    \entry{Toi}{}{}{}{Et toi, tu es américan.}

    \entry{Lui/Elle}{}{}{}{Lui, il habite à Paris. | Elle, elle est étudiante.}

    \entry{Nous}{}{}{}{Nous, nous sommes français. | "On" will also use "Nous" | e.g. Nous, on est français}

    \entry{Vous}{}{}{}{Vous, vous êtres américan.}

    \entry{Eux/Elles}{}{}{}{Eux, ils ont 20 ans.}
\end{multicols}

\begin{table}[h]
\centering
\begin{tabular}{|l|c|c|c|}
\hline
\multicolumn{1}{|c|}{\textbf{}} & \textbf{masculin singulier} & \multicolumn{1}{l|}{\textbf{féminin singulier}} & \multicolumn{1}{l|}{\textbf{pluriel}} \\ \hline
\textbf{à moi}                  & mon stylo                   & ma trousse                                      & mes stylos                            \\ \hline
\textbf{à toi}                  & ton                         & ta                                              & tes                                   \\ \hline
\textbf{à elle/lui}             & son                         & sa                                              & ses                                   \\ \hline
\textbf{à nous/on}              & notre                       & notre                                           & nos                                   \\ \hline
\textbf{à vous}                 & votre                       & votre                                           & vos                                   \\ \hline
\textbf{à elles/eux}            & leur                        & leur                                            & leurs                                 \\ \hline
\end{tabular}
\caption{Adjectifs Possessifs}
\label{tab:adjectifs-possessifs}
\end{table}

\subsection*{y (Unit 3)}
\begin{center}
    there, here
\end{center}
\begin{enumerate}
    \item Le pronom \textit{y} remplace un nom de \textbf{lieu}. \\
    e.g. Il va a Paris. Il y va en avion. \\
    \fbox{
        \parbox{0.9\linewidth}{
        \textbf{Attention}:
        \begin{enumerate}
            \item je $\rightarrow$ j' + y. \\
            e.g. Je vais a Paris. J'y vais.
        \end{enumerate}
        }
    }
    \item Au present de l'indicatif. Le pronom y se place entre le sujet et le verbe. (In the present indicative. The pronoun y is placed between the subject and the verb). \\
    e.g. \textbf{sujet + y + verbe}
\end{enumerate}

%----------------------------------------------------------------------------------------
%	Articles
%----------------------------------------------------------------------------------------
\section*{Articles}
\subsection*{Articles Définis (Unit 1)}
\begin{center}
   the (before the country names) 
\end{center}
\begin{multicols}{2}
    \entry{la}{}{}{}{used when \textbf{pays féminine (finished with "-e")} | e.g. \textbf{la} Russi\textbf{e}}

    \entry{le}{}{}{}{used when \textbf{pays masculin} | e.g. \textbf{le} Japon}

    \entry{les}{}{}{}{used when \textbf{pays pluriel (finished with "-s"} | e.g. \textbf{les} États-Uni\textbf{s}}

    \entry{l'}{}{}{}{used when \textbf{pays voyelle} | e.g. \textbf{l' A}rgentine | It has the highest priority}
\end{multicols}

\subsection*{Articles Indéfinis (Unit 1)}
\begin{center}
a / some
\end{center}
\begin{multicols}{2}
    \entry{un}{}{}{a}{| used with nom masculin and singulier}

    \entry{une}{}{}{a}{| used with nom feminin and singulier}

    \entry{des}{}{}{some}{| used with pluriel}
\end{multicols}
\fbox{
\parbox{0.9\linewidth}{
    \textbf{Attention}:
    \begin{enumerate}
        \item A la forme négative, "un, une, des" $\rightarrow$ "de" (or "d'" devant une voyelle ou un "h") \newline
        J'ai \textbf{des} stylos $\rightarrow$ Je \underline{n}'ai \underline{pas} \textbf{de} stylos. \newline
        J'ai \textbf{un} ordinateur. $\rightarrow$ Je n'ai pas \textbf{d'}\underline{o}rdinateur. \newline
        Elle a \textbf{une} valise $\rightarrow$ Elle n'a pas \textbf{de} valise. \newline
        Sauf (Safe): C'est \textbf{un} stylo. $\rightarrow$ Ce n'est pas \textbf{un} stylo.
    \end{enumerate}
    }
}

\subsection*{Articles Partitifs (Unit 4)}
\begin{center}
    On utilise un \textbf{article partitif} quand on ne peut pas compter (We use a \textbf{partitive article} when we \textbf{cannot count})
\end{center}
\begin{multicols}{2}
    \entry{du}{}{}{}{masculin singulier}

    \entry{de la}{}{}{}{féminin singulier}

    \entry{de l'}{}{}{}{devant \textit{a, e, i, o, u} et \textit{h}}
\end{multicols}
\fbox{
\parbox{0.9\linewidth}{
    \textbf{Attention}:
    \begin{enumerate}
        \item Dans une phrase \textbf{négative}, \textit{du, de la, de l'} $\rightarrow$ \textit{de} ou \textit{d'} \newline
        e.g. Je \underline{ne} mange \underline{pas} \textbf{de} pain et je \underline{ne} bois \underline{pas} \textbf{d'}eau.
    \end{enumerate}
}
}

\subsection*{L'article contracté et les préposition (Unit 2)}
\begin{center}
    L'article contracté est une \textbf{préposition} (à, de ...) + un \textbf{article défini} (le, les).
\end{center}
\begin{table}[h]
\centering
\begin{tabular}{|l|l|l|l|}
\hline
                                                                                            & + \textbf{nom masculin}                                                                 & + \textbf{nom féminin}                                                                           & + \textbf{nom pluriel}                                                                     \\ \hline
à                                                                                           & au                                                                             & à la                                                                                    & aux                                                                               \\ \hline
de                                                                                          & du                                                                             & de la                                                                                   & des                                                                               \\ \hline
\begin{tabular}[c]{@{}l@{}}en face de, loin\\ de, à côté de,\\ à gauche de ...\end{tabular} & \begin{tabular}[c]{@{}l@{}}en face du,\\ loin du,\\ à côté du ...\end{tabular} & \begin{tabular}[c]{@{}l@{}}en face de la,\\ loin de la,\\ à côté de la ...\end{tabular} & \begin{tabular}[c]{@{}l@{}}en face des,\\ loin des,\\ à côté des ...\end{tabular} \\ \hline
\end{tabular}
\caption{L'article contracté}
\label{tab:article-contracte}
\end{table}

\fbox{
    \parbox{0.9\linewidth}{
        \textbf{Attention}: 
        \begin{enumerate}
            \item Devant un nom masculin singulier qui commence par une voyelle ou un "h", on utilise "à l'" ou "de l'". \newline
            Je vais \textbf{à l'}\underline{h}ôpital. Je fais \textbf{de l'}\underline{a}thlétisme. C'est en face \textbf{de l'}\underline{h}ôpital.
            \item Usually, we use "de la/du/des/de l'" when we use the verb "faire" \newline
            And we use "à la/au/aux/à l'" when we use the verbe "aller".
            \item The preposition "à..." means like \textbf{to, at} while the "de ..." means "of" when used as a preposition or "any/some" when used as an adjective. Also, "Du" is used as a partitive article, along with "de la", "de l'", to express uncountable quantities or when the amount is unknown.
            \item One thing worth noting is that when express uncountable noun, there is no \textbf{pluriel} because if it is pluriel, it must be countable right. So we can use \textbf{des} from the "un/une/des" group.
            \item Les prepositions de lieu (e.g. \textit{en face de} $\cdots$) that must have \textbf{de} behind follows the same rule as "du/de la/des/de l'"
        \end{enumerate}
    }
}

\subsection*{Table that summarises Articles in French}
\begin{table}[h]
\centering
\begin{tabular}{|l|l|l|l|}
\hline
\multicolumn{1}{|c|}{} & \begin{tabular}[c]{@{}l@{}}définis\\ (spécifique)\end{tabular}                                                                             & \begin{tabular}[c]{@{}l@{}}indéfinis\\ (général)\end{tabular}                                                                                    & \begin{tabular}[c]{@{}l@{}}partitifs\\ (on ne peut pas compter)\\ (we can't count)\end{tabular}                                        \\ \hline
masculin               & \textbf{le}                                                                                                                                         & \textbf{un}                                                                                                                                               & \textbf{du}                                                                                                                                     \\ \hline
féminin                & \textbf{la}                                                                                                                                         & \textbf{une}                                                                                                                                              & \textbf{de la}                                                                                                                                  \\ \hline
pluriel                & \textbf{les}                                                                                                                                        & \textbf{des}                                                                                                                                              &                                                                                                                                        \\ \hline
négation               &                                                                                                                                            & \textbf{de(d')}                                                                                                                                           & \textbf{de(d')}                                                                                                                                 \\ \hline
                       & \begin{tabular}[c]{@{}l@{}}Le garçon sur \textbf{la} photo\\ s'appelle Victor et \textbf{la}\\ fille, Marie. Ce sont \textbf{les}\\ enfants de Virginie.\end{tabular} & \begin{tabular}[c]{@{}l@{}}- Vous avez \textbf{des}\\ enfants?\\ - Oui, \textbf{un} garçon et\\ \textbf{une} fille. Et vous?\\ - Non, je n'ai pas\\ \textbf{d'}enfants.\end{tabular} & \begin{tabular}[c]{@{}l@{}}Le soir, en général, je\\ mange \textbf{de la} salade et\\ \textbf{du} formage mais je ne\\ mange pas \textbf{de} viande.\end{tabular} \\ \hline
\end{tabular}
\caption{Articles}
\label{tab:articles}
\end{table}

%----------------------------------------------------------------------------------------
%	Prepositions
%----------------------------------------------------------------------------------------
\section*{Préposition}

\subsection*{Les préposition devant le lieu (Unit 2)}
\begin{center}
    We use \textit{à la/au/à l'/aux} plus the place. See above for more information.
\end{center}

\subsection*{Les préposition devant les villes et pays (Unit 3)}
\begin{enumerate}
    \item Les préposition \textbf{à, en, au, aux} indiquent le lieu où l'on est $\downarrow$ ou le lieu où l'on va $\rightarrow$. (The prepositions \textbf{à, en, au, aux} indicate the place where we are in now $\downarrow$ is or the place where we are going to $\rightarrow$). \newline
    \textbf{For example}, \newline
    $\downarrow$ J'habite \textbf{à} Paris, \textbf{en} France. \newline
    $\rightarrow$ Je vais \textbf{au} Maroc.
    \item Les préposition \textbf{du, de, des} indiquent le lieu d'où l'on vient. $\leftarrow$. (The preposotions \textbf{du, de, des} indicate the place where we come from $\leftarrow$) \newline
    \textbf{For example}, \newline
    $\leftarrow$ Je viens \textbf{de} Paris.
    \item Ces préposition changent selon le \textbf{genre} (masculin ou féminin) et le \textbf{nombre} (singulier ou pluriel) du nom de lieu. (These prepositions change according to the gender (masculine or feminine) and the number (singular or plural) of the place name.)
    \begin{table}[h]
    \centering
    \begin{tabular}{|l|c|c|}
    \hline
    +ville                    & \textbf{à}   & \textbf{de/d'} \\ \hline
    + pays féminin singulier  & \textbf{en}  & \textbf{de/d'} \\ \hline
    + pays masculin singulier & \textbf{au}  & \textbf{du}    \\ \hline
    + pays/île pluriel        & \textbf{aux} & \textbf{des}   \\ \hline
    \end{tabular}
    \caption{Les préposition devant les villes et pays}
    \label{tab:les-prepositions-devant-les-villes-et-pays}
    \end{table}
    \\
    \fbox{
    \parbox{0.9\linewidth}{
    \textbf{Attention}:
    \begin{enumerate}
        \item On utilise \textbf{en} devant un nom de pays (masculin ou féminin) qui commence par une voyelle (We use \textbf{en} front of a country name (masculine or feminine) which begins with a vowel). \\
        J'habite \textbf{en} \underline{I}rak. Je pars \textbf{en} \underline{A}ustralie.
    \end{enumerate}
    }
    }
\end{enumerate}

\section*{Les questions (Unit 3)}
\subsection*{est-ce que}
\begin{enumerate}
    \item \textbf{est-ce que} + sujet + verbe $\cdots$? $\rightarrow$ Réponse fermée: oui/non \\
    For example, \textbf{Est-ce que} tu aimes Paris? \textbf{Oui}, j'aime beaucoup cette ville. \\
    \fbox{
        \parbox{0.9\linewidth}{
        \textbf{Attention}:
        \begin{enumerate}
            \item est-ce que $\rightarrow$ est-ce qu' + sujet qui commence par une voyelle ou un h. \\
            e.g. \textbf{Est-ce qu'}elle aime Paris? \textbf{Est-ce qu'}Henri est italien?
        \end{enumerate}
        }
    }
\end{enumerate}

\subsection*{Qu'est-ce que}
\begin{enumerate}
    \item \textbf{Qu'est-ce que} + sujet + verbe $\cdots$? = Sujet + verbe + \textbf{quoi}? $\rightarrow$ Réponse ouverte. \\
    For example, \textbf{Qu'est-ce que} tu fais à Paris? Je visite des musée, je vais au restaurant. \\
    \fbox{
        \parbox{0.9\linewidth}{
        \textbf{Attention}:
        \begin{enumerate}
            \item que dans une question inversee (que in a reverse question). e.g. \textbf{Que} fais-tu à Paris?
            \item quoi dans une question avec intonation montante (quoi in a question with rising intonation). e.g. Tu fais \textbf{quoi} à Paris?
        \end{enumerate}
        }
    }
\end{enumerate}

\section*{Adjectif (Unit 3)}
\begin{enumerate}
    \item L'adjectif s'accorde en genre (masculin ou feminin) avec le nom (the adjective agrees in gender (masculine or feminine) with the noun)
    \begin{table}[h]
    \centering
    \begin{tabular}{|l|l|l|}
    \hline
    \textbf{masculin}       & \textbf{feminin} & \textbf{exemples}                                                                                           \\ \hline
    a,i,o,u, une consonne & +-e              & \begin{tabular}[c]{@{}l@{}}Il est jol\underline{i}, Elle est joli\textbf{e}\\ Il est gourman\underline{d}. Elle est gourmand\textbf{e}.\end{tabular}  \\ \hline
    -e                      & \O             & Il est sympathique. Elle est sympathique                                                                    \\ \hline
    -ien/-on                & +-ne             & \begin{tabular}[c]{@{}l@{}}Il est ital\underline{ien}. Elle est ital\textbf{ienne}\\ Il est b\underline{on}. Elle est b\textbf{onne}.\end{tabular}    \\ \hline
    -(i)er                  & -(i)ère         & \begin{tabular}[c]{@{}l@{}}Il est étrang\underline{er}. Elle est étrang\textbf{ère}.\\ Il est f\underline{ier}. Elle est f\textbf{ière}.\end{tabular} \\ \hline
    -eux                    & -euse            & Il est chaleur\underline{eux}. Elle est chaleur\textbf{euse}.                                                                    \\ \hline
    \end{tabular}
    \caption{}
    \label{tab:adjectif}
    \end{table}
    \\
    \fbox{
        \parbox{0.9\linewidth}{
        \textbf{Attention}:
        \begin{enumerate}
            \item beau $\rightarrow$ belle. gros $\rightarrow$ grosse
        \end{enumerate}
        }
    }
    \item L'adjectif s'accorde en nombre (singulier ou pluriel) avec le nom
    \begin{table}[h]
    \centering
    \begin{tabular}{|l|l|l|}
    \hline
    \textbf{singulier}           & \textbf{pluriel} & \textbf{exemples}                                                                                                         \\ \hline
    -une voyelle, une consonne & +-s              & \begin{tabular}[c]{@{}l@{}}Il est sympathique. Ils sont sympathique\textbf{s}.\\ Il est gourmand. Ils sont gourmand\textbf{s}.\end{tabular} \\ \hline
    -s, -x                       & \O             & \begin{tabular}[c]{@{}l@{}}Il est gros. Ils sont gros.\\ Il est chaleureux. Ils sont chaleureux.\end{tabular}             \\ \hline
    -eau                         & +-x              & Il est b\underline{eau}. Ils sont b\textbf{eaux}.                                                                                              \\ \hline
    -al                          & $\rightarrow$-aux             & Il est génial. Ils sont géni\textbf{aux}.                                                                                          \\ \hline
    \end{tabular}
    \caption{}
    \label{tab:adjectif-nombre}
    \end{table} 
\end{enumerate}

\section*{Adjectifs démonstratifs (Unit 4)}
\begin{center}
    This/That, These/Those
\end{center}
\begin{enumerate}
    \item L'adjectif démonstratif montre quelqu'un ou quelque chose. (The demonstrative adjective shows someone or something) \newline
    \textbf{Ce} garçon est très grand! Ne mange pas \textbf{ces} bonbons!
    \item L'adjectif s'accorde en \textbf{genre} (masculin ou féminin) et en \textbf{nombre} (singulier ou pluriel) avec le nom.
    \begin{multicols}{2}
        \entry{ce}{}{}{}{masculin singulier, e.g. ce soir, cet après-midi}

        \entry{cette}{}{}{}{féminin singulier, e.g. cette année}

        \entry{ces}{}{}{}{masculin/féminin pluriel}
    \end{multicols}
    e.g. \textbf{Cette} fille est jolie. (féminin, singulier). \textbf{Ces} fille\underline{s} sont jolies. (féminin, pluriel)
\end{enumerate}
\fbox{
\parbox{0.9\linewidth}{
\textbf{Attention}:
\begin{enumerate}
    \item Devant une voyelle ou un \textit{h}, \textit{ce} $\rightarrow$ \textit{cet}. \newline
    e.g. cet \underline{e}nfant, cet \underline{h}'ôpital.
\end{enumerate}
}
}

%----------------------------------------------------------------------------------------
%	Adverb
%----------------------------------------------------------------------------------------
\section*{Adverbes (Unit 4)}
\subsection*{La place des adverbes de fréquence}
\begin{center}
    The place of adverbs of frequency
\end{center}
\begin{enumerate}
    \item L'adverbe de fréquence donne une précision sur le verbe. (The adverb of frequency gives precision on the verb). \newline
    e.g. Il mange \underline{souvent}. (He eats \underline{often})
    \item Dans une phrase \textbf{affirmative}, l'adverbe de fréquence se place \textbf{après le verbe}. (In an affirmative sentence, the adverb of frequency is placed \textbf{after the verb}. \newline
    e.g. sujet + verbe + \textbf{adverbe de fréquence} \newline
    Il sort \textbf{rarement}. Il court \textbf{souvent}. Il skie \textbf{toujours}. Je fair \textbf{parfois} du sport.
    \item Dans une phrase \textbf{négative}, l'adverbe de fréquence se place \textbf{après pas}. (In a \textbf{negative} sentence, the adverb of frequency is placed \textbf{after not}.) \newline
    e.g. sujet + ne + verbe + pas + \textbf{adverbe de fréquence} \newline
    Il \textbf{ne} court \textbf{pas souvent}. Je \textbf{n'}ai \textbf{jamais} cours le matin.
\end{enumerate}
\fbox{
\parbox{0.9\linewidth}{
    \textbf{Attention}:
    \begin{enumerate}
        \item Avec \textit{jamais}, il n'y a pas de \textit{pas} (With \textit{jamais}, there is no \textit{pas}): sujet + ne + verbe + jamais. \newline
        e.g. Il \textbf{ne} sort \textbf{jamais}.
        \item à la forme négative, rarement n'est pas utilsé (in the negative form, \textit{rarement(rarely)} is not used).
    \end{enumerate}
}
}

%----------------------------------------------------------------------------------------
%	Pronunciation
%----------------------------------------------------------------------------------------
\section*{Pronunciation}
\subsection*{[g] or [\textyogh]}
\begin{enumerate}
    \item g+a, g+o, g+u $\rightarrow$ [g]
    \item g+e, g+i, g+y $\rightarrow$ [\textyogh]
\end{enumerate}

%----------------------------------------------------------------------------------------
%	Verb
%----------------------------------------------------------------------------------------
\section*{Verb}

\subsection*{-er}
\subsubsection*{s'appeler}
\centerline{\small call oneself}
\begin{multicols}{2}
    \entry{Je m'appell\underline{e}}{}{}{I call myself}{}

    \entry{Tu t'appell\underline{es}}{}{}{You call yourself}{}

    \entry{Il/Elle s'appell\underline{e}}{}{}{He/She calls him/herself}{}

    \entry{Nous nous appe\underline{lons}}{}{}{}{We call ourselves}

    \entry{Vous vous appe\underline{lez}}{}{}{You call yourselves}{| mostly used for more than one person or one \textbf{formal} personne}

    \entry{Elles/Ils s'appel\underline{lent}}{}{}{}{They call themselves}
\end{multicols}

\subsubsection*{parler}
\centerline{\small to speak}
\begin{multicols}{2}
    \entry{Je parl\underline{e}}{}{}{I call myself}{}

    \entry{Tu parl\underline{es}}{}{}{You call yourself}{}

    \entry{Il/Rlle parl\underline{e}}{}{}{He/She calls him/herself}{}

    \entry{Nous nous par\underline{lons}}{}{}{}{We call ourselves}
    
    \entry{Vous vous par\underline{lez}}{}{}{You call yourselves}{| mostly used for more than one person or one \textbf{formal} personne}

    \entry{Elles/Ils par\underline{lent}}{}{}{}{They call themselves}
\end{multicols}

\subsubsection*{aller}
\centerline{\small to go}
\begin{multicols}{2}
    \entry{Je vai\underline{s}}{}{}{I go/I am going}{}

    \entry{Tu va\underline{s}}{}{}{You go/You are going}{}

    \entry{Il/Elle/On va}{}{}{He/She/We go(es)/is(are) going}{}

    \entry{Nous al\underline{lons}}{}{}{We go/We are going}{}

    \entry{Vous al\underline{lez}}{}{}{You go/You are going}{| mostly used for more than one person or one \textbf{formal} personn}

    \entry{Ils/Elles v\underline{ont}}{}{}{They go/They are going}{}
\end{multicols}

\subsubsection*{manger (Unit 4)}
\centerline{\small to eat}
\begin{multicols}{2}
    \entry{Je mang\underline{e}}{}{}{I eat}{}

    \entry{Tu mang\underline{es}}{}{}{You eat}{}

    \entry{Il/Elle/On mang\underline{e}}{}{}{He/She/We eat(s)}{}

    \entry{Nous mange\underline{ons}}{}{}{We eat}{}

    \entry{Vous mang\underline{ez}}{}{}{You eat}{| mostly used for more than one person or one \textbf{formal} personn}

    \entry{Il/Elles mang\underline{ent}}{}{}{They eat}{}
\end{multicols}

\subsubsection*{acheter (Unit 4)}
\centerline{\small to buy}

\subsection*{-re}
\subsubsection*{être}
\centerline{\small be}
\begin{multicols}{2}
    \entry{Je sui\underline{s}}{}{}{I am ...}{| Je suis australien.}

    \entry{Tu e\underline{s}}{}{}{You are ...}{}

    \entry{Il/Elle es\underline{t}}{}{}{He/She is ...}{}

    \entry{Nous somm\underline{es}}{}{}{We are ...}{}

    \entry{On es\underline{t}}{}{}{We are ...}{}

    \entry{Vous ête\underline{s}}{}{}{You are ...}{| mostly used for more than one person or one \textbf{formal} personne}

    \entry{Ils/Elles son\underline{t}}{}{}{They are ...}{}
\end{multicols}

\subsubsection*{faire}
\centerline{\small to do}
\begin{multicols}{2}
    \entry{Je fai\underline{s}}{}{}{I do}{}

    \entry{Tu fai\underline{s}}{}{}{You do}{}

    \entry{Il/Elle fai\underline{t}}{}{}{He/She does}{}

    \entry{Nous fai\underline{sons}}{}{}{We do}{}

    \entry{Vous fai\underline{tes}}{}{}{You do}{| mostly used for more than one person or one \textbf{formal} personn}

    \entry{Ils/Elles f\underline{ont}}{}{}{They do}{}
\end{multicols}

\subsubsection*{prendre}
\centerline{\small to take}
\begin{multicols}{2}
    \entry{Je prend\underline{s}}{}{}{I take}{}

    \entry{Tu prend\underline{s}}{}{}{You take}{}

    \entry{Il/Elle prend}{}{}{He/She takes}{}

    \entry{Nous pren\underline{ons}}{}{}{We take}{}

    \entry{Vous pren\underline{ez}}{}{}{You take}{| mostly used for more than one person or one \textbf{formal} personn}

    \entry{Ils/Elles pren\underline{nent}}{}{}{They take}{}
\end{multicols}

\subsubsection*{connaître}
\centerline{\small to know}
\begin{multicols}{2}
    \entry{Je connai\underline{s}}{}{}{I know}{}

    \entry{Tu connai\underline{s}}{}{}{You know}{}

    \entry{Il/Elle/On connaî\underline{t}}{}{}{He/She/We know(s)}{}

    \entry{Nous connaiss\underline{ons}}{}{}{We know}{}

    \entry{Vous connaiss\underline{ez}}{}{}{You know}{| mostly used for more than one person or one \textbf{formal} personn}

    \entry{Ils/Elles connaiss\underline{ent}}{}{}{They know}{}
\end{multicols}

\subsubsection*{Boire (Unit 4)}
\centerline{\small to drink}
\begin{multicols}{2}
    \entry{Je boi\underline{s}}{}{}{I drink}{}

    \entry{Tu boi\underline{s}}{}{}{You drink}{}

    \entry{Il/Elle/On boi\underline{t}}{}{}{He/She/We drink(s)}{}

    \entry{Nous buv\underline{ons}}{}{}{We drink}{}

    \entry{Vous buv\underline{ez}}{}{}{You drink}{| mostly used for more than one person or one \textbf{formal} personn}

    \entry{Ils/Elles boiv\underline{ent}}{}{}{They drink}{}
\end{multicols}

\subsection*{-ir}
\subsubsection*{avoir}
\centerline{\small have}
\begin{multicols}{2}
    \entry{J'ai}{}{}{I have ...}{}

    \entry{Tu a\underline{s}}{}{}{You have ...}{}

    \entry{Il/Elle a}{}{}{He / She has}{}

    \entry{Nous av\underline{ons}}{}{}{We have ...}{}

    \entry{Vous av\underline{ez}}{}{}{You have ... }{| mostly used for more than one person or one \textbf{formal} personne}

    \entry{Ils/Elles on\underline{t}}{}{}{They have ...}{}
\end{multicols}

\subsubsection*{vouloir}
\centerline{\small want}
\begin{multicols}{2}
    \entry{Je veu\underline{x}}{}{}{I want}{}

    \entry{Tu veu\underline{x}}{}{}{You want}{}

    \entry{Il/Elle/On veu\underline{t}}{}{}{He/She/We want(s)}{}

    \entry{Nous vou\underline{lons}}{}{}{We want}{}

    \entry{Vous vou\underline{lez}}{}{}{You want}{| mostly used for more than one person or one \textbf{formal} personn}

    \entry{Ils/Elles v\underline{eulent}}{}{}{They want}{}
\end{multicols}

\subsubsection*{venir}
\centerline{\small to come from + de/d'/du/des}
\begin{multicols}{2}
    \entry{Je vien\underline{s}}{}{}{I come from}{}

    \entry{Tu vien\underline{s}}{}{}{you come from}{}

    \entry{Il/Elle/On vien\underline{t}}{}{}{He/She/We come(s) from}{}

    \entry{Nous ven\underline{ons}}{}{}{We come from}{}

    \entry{Vous ven\underline{ez}}{}{}{You come from}{| mostly used for more than one person or one \textbf{formal} personn}

    \entry{Ils/Elles vienn\underline{ent}}{}{}{They come from}{}
\end{multicols}

%----------------------------------------------------------------------------------------
%	Sentence Structure
%----------------------------------------------------------------------------------------
\section*{Sentence Structure}

\subsection*{Negation (Unit 1)}
\subsubsection*{ne ... pas}
\begin{multicols}{2}
    \entry{ne ... pas}{}{}{Not}{| used when the verb doesn't begin with an voyelle or an "h" | e.g. Je \textbf{ne} suis \textbf{pas} anglais.}

    \entry{n' ... pas}{}{}{Not}{| used when the verb begins with an voyelle or an "h" | e.g. Elle \textbf{n'}est \textbf{pas} anglaise. 
| Il \textbf{n}'habite \textbf{pas} en France}
\end{multicols}

\subsubsection*{ne ... plus}
\begin{multicols}{2}
    \entry{ne ... plus}{}{}{No longer}{| used with \textbf{faire} usually| e.g. Je \textbf{ne} fais \textbf{plus} de sport.}

    \entry{n' ... plus}{}{}{No longer}{| used with \textbf{faire} usually and verbe avec voyelle/h.| e.g. Le stylo \textbf{n'}est \textbf{plus} sur la table.}
\end{multicols}

\subsubsection*{ne ... jamais}
\begin{multicols}{2}
    \entry{ne ... jamais}{}{}{}{e.g. Je \textbf{ne} fais \textbf{jamais} de sport.}

    \entry{n' ... jamais}{}{}{}{used with verbe avec voyelle/h.}
\end{multicols}

\subsubsection*{ne ... rien}
\begin{multicols}{2}
    \entry{ne ... rien}{}{}{}{e.g. Je \textbf{ne} fais \textbf{rien}.}

    \entry{n' ... rien}{}{}{}{used with verbe avec voyelle/h.}
\end{multicols}

\fbox{
    \parbox{0.9\linewidth}{
    \textbf{Attention}:
    \begin{enumerate}
        \item de la, du, des $\rightarrow$ de in negation\\
        Je mange \textbf{des} bananes. $\rightarrow$ Je \underline{ne} mange \underline{pas} \textbf{de} bananes.
    \end{enumerate} 
    }
}

%----------------------------------------------------------------------------------------
%	Regular Vocabulary
%----------------------------------------------------------------------------------------
\section*{Regular Vocabulary}
\begin{center}
    The way to remember whether a noun is feminin or masculin: Remember it's \textit{un ...} or \textit{une ...}, \textit{le} or \textit{la}
\end{center}

\subsection*{Unit 1}
\subsubsection*{Date}
\begin{multicols}{2}
    \entry{lundi}{}{}{Monday}{}

    \entry{mardi}{}{}{Tuesday}{}

    \entry{mercredi}{}{}{Wednesday}{}

    \entry{jeudi}{}{}{Thursday}{}

    \entry{vendredi}{}{}{Friday}{}

    \entry{samedi}{}{}{Saturday}{}

    \entry{dimanche}{}{}{Sunday}{}
\end{multicols}

\subsubsection*{Month}
\begin{multicols}{2}
    \entry{janvier}{}{}{January}{}

    \entry{février}{}{}{February}{}

    \entry{mars}{}{}{March}{}

    \entry{avril}{}{}{April}{}

    \entry{mai}{}{}{May}{}

    \entry{juin}{}{}{June}{}

    \entry{juillet}{}{}{July}{}

    \entry{août}{}{}{August}{}

    \entry{septembre}{}{}{September}{}

    \entry{octobre}{}{}{October}{}

    \entry{novembre}{}{}{November}{}

    \entry{décembre}{}{}{December}{}
\end{multicols}

\subsubsection*{Number (1-69)}
\begin{center}
    1-10
\end{center}
\begin{multicols}{2}
    \entry{zéro}{}{}{zero / 0}{}

    \entry{un}{}{}{one / 1}{}

    \entry{deux}{}{}{two / 2}{}

    \entry{trois}{}{}{three / 3}{}

    \entry{quatre}{}{}{four / 4}{}
    
    \entry{cinq}{}{}{five / 5}{}

    \entry{six}{}{}{six / 6}{}

    \entry{sept}{}{}{seven / 7}{}

    \entry{huit}{}{}{eight / 8}{}

    \entry{neuf}{}{}{nine / 9}{}

    \entry{dix}{}{}{ten / 10}{}
\end{multicols}


\begin{center}
    11-19
\end{center}
\begin{multicols}{2}
    \entry{onze}{}{}{eleven / 11}{} 

    \entry{douze}{}{}{twelve / 12}{}

    \entry{treize}{}{}{thirteen / 13}{}

    \entry{quatorze}{}{}{fourteen / 14}{}

    \entry{quinze}{}{}{fifteen / 15}{}

    \entry{seize}{}{}{sixteen / 16}{}

    \entry{dix-sept}{}{}{seventeen / 17}{}

    \entry{dix-huit}{}{}{eighteen / 18}{}

    \entry{dix-neuf}{}{}{nineteen / 19}{}
\end{multicols}

\begin{center}
    20-29
\end{center}
\begin{multicols}{2}
    \entry{vingt}{}{}{twenty / 20}{}
    
    \entry{vingt et un}{}{}{twenty one / 21}{}
    
    \entry{vingt deux}{}{}{twenty two / 22}{}
    
    \textit{The rest just use \textbf{vingt} plus number from 3-9.}
\end{multicols}

\begin{center}
    30-39
\end{center}
\begin{multicols}{2}
    \entry{trente}{}{}{thirty / 30}{}
    
    \entry{trente-et-un}{}{}{thirty one / 31}{}
    
    \textit{Follow what \textbf{vingt} does.}{}
\end{multicols}

\begin{center}
    40-49
\end{center}
\begin{multicols}{2}
    \entry{quarante}{}{}{forty / 40}{}

    \textit{Follow what \textbf{trente} does.}
\end{multicols}

\begin{center}
    50-59
\end{center}
\begin{multicols}{2}
    \entry{cinquante}{}{}{fifty / 50}{}

    \textit{Follow what \textbf{quarante} does.}
\end{multicols}

\begin{center}
    60-69
\end{center}
\begin{multicols}{2}
    \entry{soixante}{}{}{sixty / 60}{}

    \entry{soixante-neuf}{}{}{sixty nine / 69}{}
\end{multicols}

\subsubsection*{Country and Nationality}
\begin{center}
    Pays
\end{center}
\begin{multicols}{2}
    \entry{la Chine}{}{}{}{China}

    \entry{la France}{}{}{}{France}

    \entry{la Hongrie}{}{}{}{Hungry}

    \entry{l'Allemagne}{}{}{}{Germany}

    \entry{l'Italie}{}{}{}{Italy}

    \entry{les États-Unis}{}{}{}{United States}

    \entry{le Danemark}{}{}{}{Denmark}

    \entry{le Luxembourg}{}{}{}{Luxembourg}

    \entry{le Gabon}{}{}{}{Gabon}
\end{multicols}
\fbox{
    \parbox{0.9\linewidth}{
    \textbf{Attention}:
    \begin{enumerate}
        \item Ces pays sont masculins: \textit{le Bélize, le Cambodge, le Mexique, le Mozambique, le zaïre et le Zimbabwe.}
    \end{enumerate}
    }
}

\begin{center}
    Nationality
\end{center}
\begin{multicols}{2}
    \entry{francais(e)}{}{}{French}{}

    \entry{anglais(e)}{}{}{English}{}

    \entry{japonais(e)}{}{}{Japanese}{}

    \entry{pakistanais(e)}{}{}{Pakistani}{}

    \entry{néo-zélandais(e)}{}{}{New Zealander}{}

    \entry{polonais(e)}{}{}{Polish}{}

    \entry{singapourien(ne)}{}{}{Singaporean}{}

    \entry{australien(ne)}{}{}{Australian}{}

    \entry{canadien(ne)}{}{}{Canadien}{}

    \entry{cambodgien(ne)}{}{}{Cambodian}{}

    \entry{malaisien(ne)}{}{}{Malaisian}{}

    \entry{vietnamien(ne)}{}{}{Vietnamese}{}

    \entry{indien(ne)}{}{}{Indian}{}

    \entry{américain(e)}{}{}{American}{}

    \entry{suédois(e)}{}{}{Swedish}{}

    \entry{chinois(e)}{}{}{Chinese}{}

    \entry{espagnol(e)}{}{}{Spanish}{}

    \entry{allemand(e)}{}{}{German}{}

    \entry{russe}{}{}{Russian}{}

    \entry{belge}{}{}{Belgian}{}
\end{multicols}

\begin{center}
    Change rules for nationality
\end{center}
\begin{table}[h]
\centering
\begin{tabular}{|lll|}
\hline
\multicolumn{1}{|c|}{\textbf{masculin}} & \multicolumn{1}{c|}{\textbf{féminin}} & \textbf{exemples}                                                                                                                                                                                             \\ \hline
\multicolumn{1}{|l|}{-e}                & \multicolumn{1}{l|}{same}             & \begin{tabular}[c]{@{}l@{}}Il est belge. Elle est belge.\\ Il est russe. Elle est russe.\end{tabular}                                                                                                         \\ \hline
\multicolumn{1}{|l|}{-s}                & \multicolumn{1}{l|}{+-e}              & \begin{tabular}[c]{@{}l@{}}Il est français. Elle est française.\\ Il est suédois. Elle est suédoise.\\ Il est japonais. Elle est japonaise.\end{tabular}                                                      \\ \hline
\multicolumn{1}{|l|}{-d}                & \multicolumn{1}{l|}{+-e}              & Il est allemand. Elle est allemande.                                                                                                                                                                          \\ \hline
\multicolumn{1}{|l|}{-n}                & \multicolumn{1}{l|}{+-e}              & \begin{tabular}[c]{@{}l@{}}Il est argentin. Elle est argentine.\\ Il est mexicain. Elle est mexicaine.\end{tabular}                                                                                           \\ \hline
\multicolumn{1}{|l|}{-ien}              & \multicolumn{1}{l|}{+-ne}             & \begin{tabular}[c]{@{}l@{}}Il est colombien. Elle est colombienne.\\ Il est péruvien. Elle est péruvienne.\\ Il est équatorien. Elle est équatorienne.\\ Il est norvégien. Elle est norvégienne.\end{tabular} \\ \hline
\multicolumn{3}{|l|}{Attention: grec $\rightarrow$ grecque; turc $\rightarrow$ turque; coréen $\rightarrow$ coréenne}                                                                                                                                                                           \\ \hline
\end{tabular}
\caption{Adjectifs de Nationalité}
\label{tab:adjectifs-de-nationalite}
\end{table}

\begin{center}
    Change rules for pluriel
\end{center}
\begin{table}[h]
\centering
\begin{tabular}{|l|l|}
\hline
\multicolumn{1}{|c|}{\textbf{Pour former le pluriel}} & \multicolumn{1}{c|}{\textbf{exemples}}                                                        \\ \hline
+-s                                                   & \begin{tabular}[c]{@{}l@{}}Il est belge.\\ Ils sont belges.\\ Elles sont belges.\end{tabular} \\ \hline
si l'adjectif porte déjà un -s au singuler            & \begin{tabular}[c]{@{}l@{}}Il est chinois.\\ Ils sont chinois.\end{tabular}                   \\ \hline
\end{tabular}
\caption{Pluriel}
\label{tab:pluriel-adjectifs-de-nationalite}
\end{table}

\subsubsection*{Emotion and Feelings}
\begin{multicols}{2}
    \entry{triste}{}{a.}{sad}{| Je \textbf{suis} triste.}

    \entry{forme}{}{n.f.}{shape}{Je \textbf{suis} en forme(I am healthy)}

    \entry{malade}{}{a.}{sick}{Je \textbf{suis} malade.}

    \entry{fatigué}{}{a.}{tired}{| Je \textbf{suis} fatigué.}
    
    \entry{soif}{}{n.f.}{thirst}{| J'\textbf{ai} soif.}
    
    \entry{chaud}{}{n.m.}{hot}{| J'\textbf{ai} chaud.}

    \entry{sommeil}{}{n.m.}{sleep}{J'\textbf{ai} sommeil.}

    \entry{froid}{}{a.}{cold}{J'\textbf{ai} froid.}

\end{multicols}

\subsubsection*{Profession}
\begin{multicols}{2}
    \entry{un(e) pâtissier(-ière)}{pah-tees-yay}{n.m.}{pastry chef}{| Pierre Hermé est pâtissier.}

    \entry{un(e) chanteur(-se)}{shahn-tuhr}{n.}{singer}{| Stromae est chanteur.}

    \entry{un(e) nageur(-se)}{nah-zhur}{n.}{swimmer}{| Camillie Lacourt est nageur.}

    \entry{un(e) journaliste}{zhoor-nah-leest}{n.}{journalist}{| Léa Salamé est journaliste}

    \entry{un(e) architecte}{}{}{architect}{}

    \entry{un(e) footballeur(-se)}{}{}{footballer}{| Kylian Mbappé est footballeur.}

    \entry{un(e) acteur(-trice)}{}{}{actor / actress}{}

    \entry{un(e) photographe}{}{}{photographer}{}
\end{multicols}

\subsubsection*{Unclassfied}
\begin{multicols}{2}
    \entry{mot}{moh}{n.m.}{word}{| Vous connaissez quels mots?(What words do you know)}

    \entry{phrase}{frahz}{n.f.}{sentence}{| phrase simple(simple sentences)}

    \entry{majuscule}{mah-zhoo-skool}{n.f.}{capital letters}{} 
\end{multicols}

\subsection*{Unit 2}
\subsubsection*{Objets}
\begin{multicols}{2}
    \entry{un stylo}{}{n.m.}{a pen}{}

    \entry{des ciseaux}{}{n.}{scissors}{}

    \entry{un cahier}{}{n.m.}{a notebook}{}

    \entry{une trousse}{}{n.f.}{a small bag}{| small like a pencil bag}

    \entry{une gomme}{}{n.f.}{an erasor}{}

    \entry{des clés}{}{n.}{keys}{}

    \entry{une clé}{}{n.f.}{a key}{}

    \entry{des lunettes}{}{n.}{glasses}{}

    \entry{un livre}{}{n.m.}{a book}{}

    \entry{un tableau}{}{n.m.}{a blackboard}{}

    \entry{un chaise}{}{n.m.}{a chair}{}

    \entry{un ordinateur}{}{n.m.}{a computer}{}

    \entry{un téléphone}{}{n.m.}{a telephone}{}

    \entry{un ballon}{}{n.m.}{a ball}{}

    \entry{une souris}{}{n.f.}{a mouse}{}

    \entry{une tasse}{}{n.f.}{a glass}{}

    \entry{une glace}{}{n.f.}{an ice-cream}{}
\end{multicols}

\subsubsection*{Sports}
\begin{multicols}{2}
    \entry{la danse}{}{n.f.}{dance}{}

    \entry{la natation}{}{n.f.}{swim}{| \textbf{-tion} indicates the noun is féminin}

    \entry{le jogging}{}{n.m.}{jog}{}

    \entry{la randonnée}{}{n.f.}{hiking}{}

    \entry{le ski}{}{n.m.}{ski}{}

    \entry{le VTT}{}{n.m.}{mountain bike}{| VTT is vélo tout terrain}

    \entry{le basket-ball}{}{n.m.}{basketball}{}

    \entry{le football}{}{n.m.}{football}{}

    \entry{l'athlétisme}{}{n.m.}{track}{| \textbf{-isme} indicates the nouns is masculin}
\end{multicols}

\subsubsection*{Forms}
\begin{center}
    shapes
\end{center}
\begin{multicols}{2}
    \entry{un carré}{}{n.m.}{square}{}

    \entry{un triangle}{}{n.m.}{triangle}{}

    \entry{un rond}{}{n.m.}{circle}{}

    \entry{un rectangle}{}{n.m.}{rectangle}{}
\end{multicols}

\subsubsection*{Des lieux de loisirs}
\begin{center}
    Places of Leisure
\end{center}
\begin{multicols}{2}
    \entry{la bibliothèque}{}{n.f.}{library}{}

    \entry{le cinéma}{}{n.m.}{cinema}{}

    \entry{le théâtre}{}{n.m.}{theatre}{}

    \entry{la piscine}{}{n.f.}{swimming pool}{}

    \entry{la patinoire}{}{n.f.}{ice skating rink}{}

    \entry{le restaurant}{}{n.m.}{restaurant}{}
\end{multicols}

\subsubsection*{Des Loisirs}
\begin{center}
    Leisure Activities
\end{center}
\begin{multicols}{2}
    \entry{la lecture}{}{n.f.}{read}{}

    \entry{le cinéma}{}{n.m.}{cinema}{}

    \entry{la peinture}{}{n.f.}{pain}{}

    \entry{la musique}{}{n.f.}{music}{}

    \entry{le dessin}{}{n.m.}{draw}{}

    \entry{le bricolage}{}{n.m.}{DIY}{}

    \entry{la couture}{}{n.f.}{sew}{}

    \entry{la sculpture}{}{n.f.}{sculpture}{}
\end{multicols}

\subsubsection*{Unclassfied}
\begin{multicols}{2}
    \entry{goût}{goo}{n.m.}{taste/hobbies}{| la langue est l'organe du goût(the tongue is the organ of taste)}

    \entry{loisir}{lwah-zeer}{n.m.}{leisure}{| Quels sont tes loisirs?(What are your leisure activities)}

    \entry{souligner}{}{v.t.}{underline}{}

    \entry{entourer}{}{v.t.}{circle}{}

    \entry{surligner}{}{v.t.}{highlight}{}

    \entry{dans}{}{prep.}{in}{| Dans ton sac (In your bag)}

    \entry{sculpture}{}{n.f.}{sculpture}{| \textbf{la} sculpture}

    \entry{patinoire}{}{n.f.}{ice skating rink}{| \textbf{la} patinoire}

    \entry{bricolage}{}{n.m.}{DIY}{| \textbf{le} bricolage| Je fait du bricolage.}

    \entry{piscine}{}{n.f.}{swimming pool}{| \textbf{la} piscine| Victor va à la piscine}

    \entry{restaurant}{}{n.m.}{restaurant}{| \textbf{le} restaurant}

    \entry{lieu}{}{n.m.}{place}{| Its pluriel is \textbf{lieux}}

    \entry{regarder}{}{v.t.}{watch, look}{}

    \entry{rendez-vous}{}{n.m.}{appointment}{}
\end{multicols}

\subsection*{Unit 3}
\subsubsection*{Saison}
\begin{center}
    season, n. f.
\end{center}
\begin{multicols}{2}
    \entry{le printemps}{}{n. m.}{spring}{}
    
    \entry{l'été}{}{n. m.}{summer}{}

    \entry{l'automne}{}{n. m.}{automn}{}

    \entry{l'hiver}{}{n. m.}{winter}{}
\end{multicols}

\subsubsection*{Météo}
\begin{center}
    weather, the short form of \textit{météorologie}
\end{center}
\begin{multicols}{2}
    \entry{le soleil/Il fait beau}{}{}{the sun/light}{| Il fait beau = \textbf{Il y a} du soleil.}

    \entry{la pluie/Il pleut}{}{}{the rain}{| Il pluet = \textbf{Il y a} de la pluie.}

    \entry{la neige/Il neige}{}{}{the snow}{| Il neige = \textbf{Il y a} de la neige.}

    \entry{le vent/Il y a du vent}{}{}{the wind}{}

    \entry{les neuages/C'est nuageux}{}{}{the cloud}{| \textit{nuageux} is adj. and it means cloudy. \textbf{Il y a} des neuages.}

    \entry{l'orage/Il y a des éclairs}{}{}{the storm}{}

    \entry{Il fait chaud/Il fait 30 degrés}{}{}{hot}{}

    \entry{Il fait froid/Il fait moins 5 degrés}{}{}{cold}{}
\end{multicols}
\begin{center}
    Note that the after \textit{Il y a}, we use \textit{du/de la/de l'/des} because we cannot count the number of weather.
\end{center}

\subsubsection*{Des lieux de la ville}
\begin{multicols}{2}
    \entry{un musée}{}{}{a museum}{}

    \entry{le château}{}{}{the castle}{}

    \entry{la cathédrale}{}{}{the cathedral}{}

    \entry{le parc, le jardin}{}{}{the park, the garden}{}

    \entry{un bar, un restaurant}{}{}{a bar, a restaurant}{}

    \entry{une boutique, un magasin}{}{}{a shop, a store}{}

    \entry{la rue, le boulevard}{}{}{the street, the boulevard}{}
\end{multicols}

\subsubsection*{Des moyens de transport}
\begin{multicols}{2}
    \entry{a pied}{}{}{on foot}{}

    \entry{en métro}{}{}{by metro}{}

    \entry{en bus}{}{}{by bus}{}

    \entry{en voiture}{}{}{by car}{}

    \entry{en avion}{}{}{by plane}{}

    \entry{a cheval}{}{}{on horse}{}

    \entry{en bateau}{}{}{by boat}{}

    \entry{a vélo}{}{}{by bike}{}
\end{multicols}

\subsubsection*{Les prepositions de lieu}
\begin{multicols}{2}
    \entry{dans}{}{}{in}{| dans le cube}

    \entry{devant}{}{}{in front}{| devant le cube}

    \entry{derrière}{}{}{behind}{| derrière le cube}

    \entry{sur}{}{}{on}{| sur le cube}

    \entry{sous}{}{}{below}{| sous le cube}

    \entry{en face (de)}{}{}{in front of}{| en face du cube}

    \entry{à côté (de)}{}{}{next to}{| à côté du cube}

    \entry{près (de)}{}{}{close to}{| près du cube}

    \entry{loin (de)}{}{}{far from}{| loin du cube}

    \entry{entre}{}{}{between}{| entre le cube \textbf{et} la colonne}

    \entry{à droite (de)}{}{}{on the right of}{| à droite du cube}

    \entry{à gauche (de)}{}{}{on the left of}{| à gauche du cube}
\end{multicols}

\subsubsection*{Demander et indiquer son chemin}
\begin{center}
    Ask and show directions
\end{center}
\begin{multicols}{2}
    \entry{aller tout droit}{}{}{go straight}{}

    \entry{tourner à gauche}{}{}{turn left}{| Vous tournez à gauche}

    \entry{tourner à droite}{}{}{turn right}{| Vous tournez à droite}

    \entry{au bout de la rue}{}{}{at the end of the road}{}
\end{multicols}

\subsubsection*{Verbs in direction}
\begin{multicols}{2}
    \entry{Je cherche/vous cherchez}{}{}{to look for}{| Vous cherchez l’office de tourisme (You are looking for the tourist office)}

    \entry{Vous sortez}{}{}{to exit}{| Vous sortez du parc de l’Artillerie (You are leaving the Artillery Park)}

    \entry{Vous prenez}{}{}{to take}{| Vous prenez la première rue à droite (You take the first street on the right)}

    \entry{Vous continuez}{}{}{to continue}{| Vous continuez tout droit (You continue straight ahead)}

    \entry{Vous tournez}{}{}{to turn}{| Au bout de la rue, vous tournez à droite (At the end of the street, turn right)}
\end{multicols}

\subsubsection*{Les nombres ordinaux}
\begin{multicols}{2}
    \entry{premier}{}{}{first}{}

    \entry{deuxième}{}{}{second}{}

    \entry{troisième}{}{}{third}{}

    \entry{quatrième}{}{}{fourth}{}

    \entry{cinquième}{}{}{fifth}{}

    \entry{sixième}{}{}{sixth}{}

    \entry{huitième}{}{}{eighth}{}

    \entry{neuvième}{}{}{ninth}{}

    \entry{dixième}{}{}{tenth}{}
\end{multicols}

\subsubsection*{Unclassfied}
\begin{multicols}{2}
    \entry{marcher}{}{v.i.}{walk}{}

    \entry{la campagne}{}{n.f.}{countryside}{}

    \entry{gourmand}{}{adj.}{}{means love eating}

    \entry{sympathique}{}{adj.}{friendly}{}

    \entry{nouveau}{}{adj.}{new}{}

    \entry{le panneau}{}{n.m.}{sign}{}

    \entry{bizarre}{}{adj.}{weird}{}

    \entry{un escalier}{}{n.m.}{stairs}{}

    \entry{longer}{}{v.t.}{to walk along}{}
\end{multicols}

\subsection*{Unit 4}
\subsubsection*{Adverbes de fréquence}
\begin{multicols}{2}
    \entry{jamais}{}{adv.}{never}{}

    \entry{rarement}{}{adv.}{rarely}{}

    \entry{parfois}{}{adv.}{sometimes}{}

    \entry{quelquefois}{}{adv.}{sometimes}{}

    \entry{régulièrement}{}{adv.}{regularly}{}

    \entry{souvent}{}{adv.}{often}{}

    \entry{toujours}{}{adv.}{always}{}

    \entry{tous les jours}{}{}{every day}{}

    \entry{tous les mois}{}{}{every month}{}

    \entry{toutes les semains}{}{}{every week}{}

    \entry{1 fois par jour}{}{}{1 time per day}{}

    \entry{2 fois par semaine}{}{}{2 times per week}{}

    \entry{3 fois par mois}{}{}{3 times per month}{}

    \entry{la semaine}{}{}{every weekend}{| If without la, that means only one weekend, same for le + date, like le lundi}
\end{multicols}

\subsubsection*{Quelques commerces}
\begin{center}
    Some shops
\end{center}
\begin{multicols}{2}
    \entry{la boucherie}{}{n.f.}{the butcher shop}{| for meat}

    \entry{la boulangerie}{}{n.f.}{the bakery}{}

    \entry{la fromagerie}{}{n.f.}{the cheese factory}{}

    \entry{l'épicerie}{}{n.f.}{the grocery store}{}

    \entry{la poissonnerie}{}{n.f.}{the fishmonger}{| for sea/fish product}

    \entry{la pâtisserie}{}{n.f.}{the pastry}{| yummier than bread}

    \entry{le marchand de fruits et légume}{}{}{the fruit and vegetable merchant}{}
\end{multicols}

\subsubsection*{Food}
\begin{multicols}{2}
    \entry{les carottes}{}{n.f.}{carrot}{}

    \entry{les oignons}{}{n.m.}{onion}{}

    \entry{le poisson}{}{n.m.}{fish}{}

    \entry{l'ail}{}{n.m.}{garlic}{}

    \entry{le sel}{}{n.m.}{salt}{}

    \entry{les asperges}{}{n.f.}{asparagus}{| a kind of vegetable}

    \entry{les champignons}{}{n.m.}{mushroom}{}

    \entry{le lait}{}{n.m.}{milk}{| du lait}

    \entry{le pain}{}{n.m.}{bread}{| du pain}

    \entry{le fromage}{}{n.m.}{cheese}{| du fromage}

    \entry{un œuf}{}{n.m.}{egg}{}

    \entry{farine}{}{n.f.}{flour}{| 250g de farine, de la farine}

    \entry{beurre}{}{n.m.}{butter}{| un peu de beurre, du beurre}

    \entry{crème}{}{n.f.}{cream}{| de la crème}

    \entry{spéculoos}{}{n.m.}{biscuits}{| des spéculoos}

    \entry{sucre}{}{n.m.}{sugar}{| 1 cuillère à soupe de sucre, du sucre}

    \entry{eau}{}{n.f.}{water}{| un peu d'eau, je bois de l'eau}

    \entry{viande}{}{n.f.}{meat}{| de la viande}

    \entry{une tomate}{}{n.f.}{tomato}{}
\end{multicols}

\subsubsection*{Les repas}
\begin{center}
    Meals
\end{center}
\begin{multicols}{2}
    \entry{le dîner}{}{v.i./n.m.}{dinner}{| Je dîne à 18 h.}

    \entry{le déjeuner}{}{v.i./n.m.}{lunch}{| Je déjeune à 13 h.}

    \entry{le petit déjeuner}{}{n.m.}{breakfast}{| Je prends le petit déjeuner à 8 h.}
\end{multicols}

\subsubsection*{Number (70-)}
\begin{center}
    70-79 (Rule: 60 + 10 $\cdots$)
\end{center}
\begin{multicols}{2}
    \entry{soixante-dix}{}{}{70}{}

    \entry{soixante et onze}{}{}{71}{}

    \entry{soixante-douze}{}{}{72}{}
\end{multicols}
\begin{center}
    80-89 (Rule: 4 * 20 + digit $\cdots$)
\end{center}
\begin{multicols}{2}
    \entry{quatre-vingts}{}{}{80}{}

    \entry{quatre-vingt-un}{}{}{81}{}

    \entry{quatre-vingt-deux}{}{}{82}{}
\end{multicols}
\begin{center}
    90-99 (Rule: 4 * 20 + 10 $\cdots$)
\end{center}
\begin{multicols}{2}
    \entry{quatre-vingt-dix}{}{}{90}{}

    \entry{quatre-vingt-onze}{}{}{91}{}

    \entry{quatre-vingt-douze}{}{}{92}{}
\end{multicols}
\begin{center}
    100-: (Rule: num * 100 + two digits)
\end{center}
\begin{multicols}{2}
    \entry{cent vingt-cinq}{}{}{125}{}

    \entry{deux cent vingt-cinq}{}{}{225}{}

    \entry{deux cent cinquante}{}{}{250}{}
\end{multicols}

\subsubsection*{Commander au restaurant}
\begin{center}
    Order at the restaurant
\end{center}
\begin{multicols}{2}
    \entry{entrée}{}{n.f.}{appetitier}{}

    \entry{plat}{}{n.m.}{dish}{}

    \entry{dessert}{}{n.m.}{dessert}{}

    \entry{formule déjeuner}{}{}{lunch formula}{}
    
    \entry{chez moi}{}{}{}{at my house}
\end{multicols}

\subsubsection*{Unclassfied}
\begin{multicols}{2}
    \entry{le grignotage}{}{n.m.}{snack}{}
\end{multicols}
%----------------------------------------------------------------------------------------
%	Phrases
%----------------------------------------------------------------------------------------
\section*{Phrases}
\subsection*{Unit 1}
\subsubsection*{S'exprimer poliment}
\begin{multicols}{2}
    \entry{Je voudrais}{voo-dray}{}{I would like}{| Je voudrais un café.(I'd like a cup of coffee}

    \entry{s'il vous plait}{}{}{please}{| Un café, s'il vous plait.(A coffee, please) | ABBR: \textbf{SVP}}

    \entry{s'il ton plait}{}{}{please}{| ABBR: \textbf{STP}}

    \entry{Merci}{}{}{Thanks}{}

    \entry{De Rien}{duh ryen}{}{It's nothing.}{}

    \entry{Pardon}{}{}{Sorry}{}

    \entry{Je vous en prie}{zhuh voo zahn pree}{}{You are welcome}{}

    \entry{Je suis désolé(e)}{day-zoh-lay(ee)}{}{I am sorry}{| \textit{désolé} is for masculin and \textit{désolée} is for feminin}
\end{multicols}

\subsection*{Unit 2}
\subsubsection*{Exprimer des goûts}
\begin{center}
    Express your hobbies
\end{center}
\begin{multicols}{2}
    \entry{J'adore}{}{}{I love...}{| J'adore chanter.(I love singing)}

    \entry{J'aime}{}{}{I like}{| J'aime \textbf{la} lecture.}

    \entry{Je n'aime pas}{}{}{I don't like}{| Je n'aime pas le sport.}

    \entry{Je déteste}{}{}{I hate}{| Je déteste danser.}
\end{multicols}

\fbox{
    \parbox{0.9\linewidth}{
        \textbf{Attention}:
        \begin{enumerate}
            \item J'adore + nom = J'adore + verbe infinitif (to do) \textit{The same applies to J'aime, Je déteste and Je n'aime} \newline
            e.g. J'adore \textbf{la} dance = J'adore \textbf{dancer}.
        \end{enumerate}
    }
}

\subsubsection*{Demander l'heure}
\begin{center}
    Ask for time
\end{center}
\begin{multicols}{2}
    \entry{Quelle heure il est?}{}{}{}{What time is it?}

    \entry{Il est quelle heure?}{}{}{}{It is what time?}

    \entry{C'est à quelle heure}{}{}{}{It is at what time?}
\end{multicols}

\subsubsection*{Indiquer l'heure}
\begin{center}
    Indicate the time
\end{center}
\begin{multicols}{2}
    \entry{Il est neuf heures.}{}{}{}{It is 9:00}

    \entry{Il est neuf hueres quinze.}{}{}{}{It is 9:15.}

    \entry{Il est neuf heures trente}{}{}{}{It is 9:30.}

    \entry{Il est neuf heures quarante-cinq}{}{}{}{It is 9:45.}

    \entry{Il est midi}{}{}{}{It is noon.}

    \entry{Il est minuit}{}{}{}{It is midnight.}
\end{multicols}

\subsubsection*{Préciser un moment}
\begin{center}
    Indicate time precisely
\end{center}
\begin{multicols}{2}
    \entry{aujourd'hui}{}{}{today}{}

    \entry{demain}{}{}{tomorrow}{}

    \entry{tôt}{}{}{early}{| C'est tôt(It's early)}

    \entry{tard}{}{}{late}{| C'est tard(It's late)}

    \entry{le matin}{}{}{morning}{}

    \entry{le soir}{}{}{evening}{}

    \entry{le week-end}{}{}{weekends}{}

    \entry{le semaine}{}{}{week days}{}

    \entry{l'après-midi}{}{}{afternoon}{| \textbf{après} means after. \textbf{midi} means noon. so means \textbf{afternoon}}
\end{multicols}

\subsubsection*{Souhaiter quelque chose à quelqu'un}
\begin{center}
    To wish something for someone
\end{center}
\begin{multicols}{2}
    \entry{Bonne journée}{}{n.f.}{Wish you a good day!}{}

    \entry{Bonne soirée}{}{n.f.}{Have a good evening!}{}

    \entry{Bonne nuit}{}{n.f.}{Have a good night!}{}

    \entry{Bonne chance}{}{n.f.}{Good luck!}{}

    \entry{Bon courage}{}{n.m.}{Good luch!}{}

    \entry{Bonne année}{}{n.f.}{Have a good year!}{| Usually used at the start of one year.}

    \entry{Bonne santé}{}{n.f.}{Wish you good health!}{}

    \entry{Bonne fête}{}{n.f.}{Happy festival}{| A French culture.}

    \entry{Bon voyage}{}{n.m.}{Have a good journey!}{}

    \entry{Bonnes vacances}{}{pl.}{Happy holidays}{}

    \entry{Joyeux anniversaire}{}{n.f.}{Happy birthday!}{}

    \entry{Joyeux Noël}{}{n.m.}{Happy Christmas}{}

    \entry{Bon appétit}{}{n.m.}{Have a good eat}{}
\end{multicols}

\subsubsection*{Unclassfied}
\begin{enumerate}
    \item \textbf{Qu'est-ce que c'est} (What is it?) \newline
    C'est utile. C'est un rectangle. (It is useful. It is a rectangle) \newline
    C'est gris. C'est une table? (It is grey. Is it a table?)
    \item \textbf{C'est quoi, ton sport?} (What is your sport?/What sport do you do?) \newline
    Mon sport, c'est le foot. Je regarde le foot avec mes amis le week-end et je joue avec mon frère. (My sport is football. I watch football with my friends on weekends and play with my brother.)
    \item \textbf{Vous faites quoi et où} (You do what and where?) \newline
    Je vais lire à la bibliothèque. (I'm going to read at the library) \newline
    Je fais du dessin le week-end. (I draw on weekends)
\end{enumerate}

\subsection*{Unit 3}
\subsubsection*{Say Temperature}
\begin{multicols}{2}
    \entry{Il fait $\cdots$ degré}{}{}{It has $\cdots$ dergee.}{| Il fait 19 degrés}

    \entry{La température est $\cdots$ degré}{}{}{The temperature is of $\cdots$ degree.}{| La température est du 9 degrés}
\end{multicols}

\subsubsection*{Exprimer une envie, un besoin}
\begin{center}
    Express a desire, a need
\end{center}
\begin{multicols}{2}
    \entry{J'ai envie de/d'}{}{}{I have a desire of}{| J'ai envie d'aller au Mexique. (I want to go to Mexico)}

    \entry{J'ai besoin de/d'}{}{}{I have a need of}{| J'ai besoin de soleil. (I need sun)}

    \entry{Je voudrais}{}{}{I would want}{| Je voudrais partir avec toi. (I would like to go with you)}

    \entry{J'aimerais}{}{}{I would like}{| J'aimerais voir des montagnes. (I would like to see mountains)}
\end{multicols}

\subsubsection*{Unclassfied}
\begin{enumerate}
    \item \textbf{Quel jour sommes-nous?} (What day is it?) \newline
    C'est mercredi (It is Wednesday.)
    \item \textbf{Il fait quel temps?/Quel temps fait-il?} (What's the weather like?) \newline
    Il fait beau. (It is sunny.) \newline
    Il y a \textbf{du} soleil. (It is sunny.)
    \item \textbf{Quelle est la température?} (What is the temperature?) \newline
    La température est \textbf{du} 19 degrés. (The temperature is 19 degrees) \newline
    Il fait 19 degrés. (The temperature is 19 degrees)
\end{enumerate}

\subsection*{Unit 4}
\subsubsection*{Donner un conseil}
\begin{center}
    Give advice
\end{center}
\begin{enumerate}
    \item Il faut dormir plus. (You need to sleep more)
    \item Tu peux faire du sport. (You can do sports)
    \item Vous pouvez faire des pauses. (You can take breaks)
\end{enumerate}

\subsubsection*{Donner son appréciation}
\begin{center}
    Give your appreciation
\end{center}
\textbf{Posotive}
\begin{enumerate}
    \item C'est très bon! (It's very good!)
    \item C'est excellent!
    \item C'est délicieux! (It's delicious!)
    \item L'accueil est chaleureux! (The welcome is warm!)
\end{enumerate}
\textbf{Negative}
\begin{enumerate}
    \item Ce n'est vraiment pas bon! (This is really not good!)
    \item Ce n'est pas assez cuit! (This is not cooked enough!)
    \item C'est trop cuit! (It's overcooked!)
    \item C'est un peu cher! (It's a bit expensive!)
    \item C'est trop cher! (It's too expensive!)
\end{enumerate}

\subsubsection*{L'expression de la quantité}
\begin{center}
    L'expression de la quantité sert à mesurer personnes, les aliments... (The expression of quantity is used to measure people, foods...)
\end{center}
\begin{table}[h]
\centering
\begin{tabular}{|l|l|}
\hline
en général:          & \begin{tabular}[c]{@{}l@{}}un \textbf{peu} de (a little bit of)\\ \textbf{beaucoup} de (a lot of)\end{tabular}                                                                                                                                                                                                                   \\ \hline
un poids             & \begin{tabular}[c]{@{}l@{}}un \textbf{gramme} (g)\\ un \textbf{kilo(gramme)} (kg)\end{tabular}                                                                                                                                                                                                                                   \\ \hline
un liquide:          & \begin{tabular}[c]{@{}l@{}}un \textbf{litre} (l)\\ un \textbf{centilitre} (cl)\\ un \textbf{millilitre} (ml)\end{tabular}                                                                                                                                                                                                                 \\ \hline
une parte d'un tout: & \begin{tabular}[c]{@{}l@{}}un \textbf{quart} (a quarter)\\ un \textbf{demi} (a half)\\ une \textbf{moitié} (a half)\end{tabular}                                                                                                                                                                                                          \\ \hline
certains aliments:   & \begin{tabular}[c]{@{}l@{}}une \textbf{pincée} de sel/poivre (a pinch of salt/pepper)\\ une \textbf{cuillère à café} de levure (a teaspoon of yeast)\\ une \textbf{cuillère à soupe} d'huile (a tablespoon of oil)\\ une \textbf{tablette} de chocolate (a bar of chocolate)\\ une \textbf{bouteille} de lait, d'eau (a bottle of milk, water)\end{tabular} \\ \hline
\end{tabular}
\caption{L'expression de la quantite}
\label{tab:expression-of-quantity}
\end{table}
\begin{enumerate}
    \item \textit{De} here means of, so it is always \textit{de/d'} (no \textit{de la, du, de l'}).
    \item Dans une recette de cuisine, il y a toujours des quantités. (In a cooking recipe, there are always quantities.) \newline
    e.g. Il faut 200 g de farine, 400 ml de lait, 3 oeufs, un peu de beurre et une cuillère à soupe d'huile.
\end{enumerate}
\fbox{
\parbox{0.9\linewidth}{
    \textbf{Attention}:
    \begin{enumerate}
        \item Devant un nom qui commence par une voyelle ou un \textit{h} muet (Before a noun begins with a vowel or a silent \textit{h}), \textit{de} $\rightarrow$ \textit{d'}. \newline
        e.g. Il achète un litre \textbf{d'}\underline{e}au et un litre \textbf{d'}\underline{h}uile.
    \end{enumerate}
}
}

\subsubsection*{Le futur proche}
\begin{enumerate}
    \item Le futur proche indique une action à venir. (The near future indicates an action to come.) \newline
    e.g. Ce week-end, je \textbf{vais} \underline{faire} du sport.
    \item Le futur proche se construit avec le verbe \textit{aller}. \newline
    sujet + \textbf{aller} au présent de l'indicatif + \textbf{verbe à l'infinitif} \newline
    e.g. Nous \textbf{allons} \underline{manger} au restaurant.
    \item à la forme \textbf{négative}, la \textbf{négation} entoure le verbe \textit{aller} \newline
    sujet + ne + aller au présent de l'indicatif + \textbf{pas} + \textbf{verbe à l'infinitif} \newline
    e.g. Il \textbf{ne} va \textbf{pas} venir avec nous.
\end{enumerate}
\fbox{
\parbox{0.9\linewidth}{
    \textbf{Attention}:
    \begin{enumerate}
        \item Devant une voyelle, \textit{ne} $\rightarrow$ \textit{n'} \newline
        e.g. Nous \textbf{n'}\underline{a}llons pas manger au restaurant.
    \end{enumerate}
}
}

\subsubsection*{Unclassfied}
\begin{multicols}{2}
    \entry{Il faut acheter quoi?}{}{}{What must he buy?}{}
\end{multicols}
\end{document}